\chapter{Background}
\label{ch:background}
\section{Chemical compounds and charges}
For the purposes of this project\footnote{Apologies to any pure chemists who find this, especially the computational kind.} a chemical compound can be defined as a grouping of elements, in various non-zero quantities. A compound may be charged or uncharged, and this depends on the charge present on each element in the compound; if the net is $0$ then the compound is considered `charge balanced' or uncharged. \cite{mcnaught1997compendium} \\

An element may have one of many possible charges, for our purposes this is a property of the element itself not its surrounding compound. For example, oxygen can be found in any of the following states: $-2,\ -1,\ +1,\ +2$. \cite{mcnaught1997compendium} \\

Loosely defined `stoichiometry' can refer to a set of elements or compounds acting as either input or output from the system, it simply relates something to a quantity. \cite{mcnaught1997compendium} \\

\section{Linear algebra}
\subsection{LU decomposition}
Lower-upper (LU) decomposition or factorisation represents the separation of a matrix into the product between a lower triangular matrix and an upper triangular matrix, with sometimes including a permutation matrix as well. It represents the matrix form of Gaussian elimination and is commonly as well as in this project, used for solving systems of linear equations.\cite{schwarzenberg1995matrix} \\
The LU decomposition utilised by Eigen in project is referred as \textit{LU factorisation with full pivoting} \cite{trefethen1997numerical} as it involves both row and column permutations.

\begin{center}
  $PAQ = LU$, \\
  \vspace{0.5em}
  where $P$ is a permutation matrix, \\
  and $Q$ is a permutation matrix that reorders the columns of $A$
\end{center}

\subsection{Kernel}
The kernel, or null-space is the set of vectors in the domain of the mapping which maps to the zero vector. When the linear map is represented as a $m \times n$ matrix the kernel is the set of solutions to the equation $AX = 0$, where $0$ is understood as the zero vector. The dimension of the kernel of $A$ is called the nullity of $A$ and is the complement to the rank of $A$.

\section{Existing approaches}
The only existing approach is done mostly manually with computers only assisting in number generations.

Generating the a viable stoichiometry is done by enumerating all possible combinations of the desired elements with one atom, and up to a desired n number of atoms, usually twenty. After all combinations are done, every solution is then charge balanced, resulting in most of the data being thrown out, resulting in a lot of wasted time and resources. \\

Arriving at a solution for desired precursor ratios and amounts is a much jarring task, as it done in a trial-and-error fashion in the lab. After each synthesis the necessary amounts are adjusted and retried until the chemists arrive at desired ratios. \\

Our proposed solution will attempt to remedy both of these setbacks, by relying on predicting stoichiometries based on already defined constraints, eliminating the need for useless post-processing of data, as well as attempting to assemble the most viable combination of precursors that would not inconvenience the chemist.\\
